%! Author = lazza
%! Date = 07/05/2022
\section{ILP limits}\label{sec:ilp-limits}
\subsection{Superscalar}\label{subsec:superscalar}
Superscalar execution allows multiples-issue and out-of-order execution.
A superscalar processor can execute more than one instruction during a clock cycle by simultaneously
dispatching multiple instructions to different execution units on the processor.
It therefore allows more throughput than would otherwise be possible at a given clock rate.
Each execution unit is not a separate processor (or a core if the processor is a multicore processor), but an
execution resource within a single CPU such as an arithmetic logic unit.
In Flynn's taxonomy, a single-core superscalar processor is classified as an SISD processor, though a single-core
superscalar processor that supports short vector operations could be classified as SIMD (single instruction stream,
multiple data streams).
A multicore superscalar processor is classified as an MIMD processor (multiple instruction streams, multiple data streams).
While a superscalar CPU is typically also pipelined, superscalar and pipelining execution are considered different
performance enhancement techniques.
The former executes multiple instructions in parallel by using multiple execution units, whereas the latter executes
multiple instructions in the same execution unit in parallel by dividing the execution (instruction) unit into
different phases.

\paragraph{Key requirements}
\begin{description}
    \item[-] Fetching more instructions per clock cycle: no major problem provided the instruction cache can sustain
    the bandwidth and can manage more requests at the same time.
    \item[-] Decide on data and control dependencies: dynamic scheduling and dynamic branch prediction
\end{description}

\paragraph{Beyond CPI = 1} 
\begin{itemize}[noitemsep]
    \item issue multiple instruction per clock-cycle
    \item varying number of instruction per cycle (1 to 8)
    \item scheduled by the compiler or HW
\end{itemize}
\[CPI_{ideal} =\frac{1}{issue-width}\]


With single-issue ILP we can be reach an ideal CPI =~1.
With superscalar execution our \textit{ideal} CPI can decrease furthermore.

\subsection{Assumptions}\label{subsec:assumptions}
The ideal machine must have:
\begin{description}
    \item[Register renaming] infinite virtual registers and all WAW and WAR hazards are avoided
    \item[Branch prediction] perfect, no mispredictions
    \item[Jump prediction] all jumps perfectly predicted, machine with perfect speculation and an unbounded buffer of
    instructions available
    \item[Memory address alias analysis]\footnote{Alias analysis is a technique used to determine if a storage location may be accessed in more than one way} addresses are know and a store can be moved before a load provided the 
    addresses not equal 
    \item[One cycle latency for all instructions] unlimited number of instructions issued per clock cycle
\end{description}

\subsection{Limits on window size}\label{subsec:limits-on-window-size}
Dynamic analysis is necessary to approach
perfect branch prediction (impossible at compile
time!).
A perfect dynamic-scheduled CPU should:
\begin{itemize}[noitemsep]
    \item[-] Look arbitrarily far ahead to find set of instructions to
    issue, predict all branches perfectly
    \item[-] Rename all registers uses (no WAW, WAR hazards);
    \item[-] Determine whether there are data dependencies
    among instructions in the issue packet, rename if
    necessary
    \item[-] Determine if memory dependencies exist among
    issuing instructions, handle them
    \item[-] Provide enough replicated functional units to allow all
    ready instructions to issue.
\end{itemize}
Size affects the number of comparisons necessary to determine
RAW dependencies.
The number of comparisons to evaluate for data dependencies among n register-to-register
instructions in the issue phase is \(n^2 - n\).
