%! Author = lazza
%! Date = 02/05/2022

\section{Exercise session 1 - Data Hazards}\label{sec:exercise-session-1}
IF instruction fetch - ID instruction decode - EX execution - MEM memory access - WB write back

single fetch per clock cycle : Microprocessor without Interlocked Pipelined Stages (SIMD)

data conflict vs data hazards?
A conflict can be an hazard

exercise1:
- highlight conflicts
- data path, introduce forwarding, memorize data path + 3 forwarding path (diagonal) + 1 forwarding in the same clock
- rescheduling

exercise2:
- highlight conflicts
Resolve data hazards
- without forwarding, stalls
- with forwarding
- scheduling
Resolve control hazards
- apply backward taken forward not taken

Problem:
solution a
\[CPI = \frac{clock cycles}{instruction} = \sum_{i=0}^{n} CPI_i \times F_i\]
where \(F_i = \frac{I_i}{Instruction count}\)

solution b,c,d vedo formulario


\section{Exercise session 2}\label{sec:exercise-session-2}
Performance analysis

Throughput vs Response time (latency?)
- faster hardware : more throughput and minor response time
- adding parallel hardware: more throughput, response time could be less, e.g, if there were a queue to serve, which
was waiting for computing resources.

Amdahl's law
1) Hardware : Enhance fraction (of our application, program), if we enhance the entire application the max overall
speedup is = 2,86.

2)Parallelism: strong scaling, #thread vs serial part of a program
Power consumption?

3) pipelining
conflict = dependency != hazard

4) dynamic branch prediction



\section{Exercise section 3}\label{sec:exercise-section-3}
a) mips pipeline
- calculate CPI, IC, MIPS
IC instruction count
MIPS million of instructions per second = Clock freq / (CPI * 10^6)

b) simple pipeline
- reschedule

c) simple pipeline
- forwarding

d) BHT 1-bit 2-bit
- colliding addresses


\section{Exercise section 4}\label{sec:exercise-section-4}

Exercise session 4

Complex Pipeline aka VLIw
- useful for FP operations (not considered in MIPS arch)
- different latency for different FU, we can reach the WB with different latency, out-of-order commit (WAW and WAR possibles)
- data and structural hazards? why structural? structural happens when two instruction need the same resource
in-order-issue in MIPS and VLIW,

FU units can be pipelined, example slide 13, DIV functional unit is multi-cycle and can't be pipeline, whereas the other
FU can be pipelined.

Complex pipeline assumption!
- in-order issue
- out-of-order commit

ese A) feasible solutions, problems:
1/4 out-of-order issue
2/4 structural hazards on c8 and c11, same resource WB is request by two instruction! (besides having data hazards)
3/4 structural hazard on C11
4/4 Feasible:
structural hazard on C11 solved by committing previous instruction and stalling 1 cycle the other.
RAW hazard solved by stalls on ISSUE stage, where we read the register (ID only decode now), IS and WB in the same cycle is OK\@.
WAR hazard instr 4 can enter the IS stage starting from cycle c7, which is WB stage (WB before or same cycle of IS) for instr 1
WAW same as before?

We cannot enter the issue stage if there's a WAR or WAW hazard until the WB stage.
Stalls for WAW and WAR need to wait in the ID stage, RAW in the IS stage, for no particular reason.

Adding buffer for instructions we stall only in the ISSUE stage letting IF and ID to not be stalled by structural hazards.

ese B) VLIW schedule
3-issue? what does that mean? maybe simultaneous issues

In clock C8 there is no WAR hazard since register 4 is read by the two instruction at the same time and then addi will overwrite
register 4  in the next instruction without compromising the other instruction.

bne instr is on cycle C10 because we need register 4 in ID stage to early evaluate the PC, so we need to leave C9 to ID.
(See exe 5, first exercise)

pipelined version of the exercise, easy.

ese C) VLIW schedule, fully pipelined

\section{Exercise session 5}\label{sec:exercise-session-5}
tiny url - exercises

Branch always taken not very useful if early prediction is present.

a) Scoreboard
See recap about scoreboard

In case of feasible configuration question, if it is not feasible one reason suffices.
MULTF and MULTD are floating point, what's the difference?

WAW hazard prevents instruction to be issued, solved at the clock cycle after previous WB
WAR hazard stuck in the issue stage, solved at the clock cycle after previous WB

b) Complex pipeline (watch it again min 52) + branch prediction
Warning: store instruction just reads registers, no transitivity: indicate all possible conflicts

infinite buffer in the issue stage, RAW and stall happens in this stage.
WAW and WAR stalls in id stage till a clock cycle before WB, in WB the IS stage can start.

target address of branch available in the fetch stage


\section{Exercise session 6}\label{sec:exercise-sessione-7}
Recall tomasulo pipeline: issue, execution, write (slide 6)

a) tomasulo v2
I1
1cc issue "1", RSi "RS1"
2cc start exe "2", Unit "LDU1"

I2
3cc cannot start execution, hazards type "RAW F6"
4cc I1 writes the result on the CDB and reservation station reads the data, next cc data is propagated to the FU

Attention: keep track of reservation station availability, I5 cannot start because of structural hazard.
Attention: two FU cannot write in the same cycle - struct CDB hazard

b) VLIW fully pipeline
VLIW is a "asap" schedule.
If there are no dependencies we can schedule two consecutive load.

Attention: the instruction with no RAW dependencies can be schedule as soon as the cycle in which the destination register
is read by a previous instruction.

Note: after the bne there waiting cycle, why?


c) scoreboard
Recall pipeline: issue, read operands, exe complete, wb

- highlight conflicts
not feasible just name one problem:
- two WB in the same cc may be struct problem (depends on the design of scoreboard, there are parallel write lines)
- out of order issue
- two IS in the same cc is a struct problem
- RAW, WAR, WAW

redo exercise