%! Author = lazza
%! Date = 03/05/2022


Exception handling

definitions
exception :
interrupt : external of internal event that needs to be processed by another program (the interrupt handler), at the end resume normal execution

External async events:
- input output device-request
- timer expiration
- power disruptions, hardware failure

Internal sync event:
- undefined opcode, privileged instruction
- arithmetic overflow, FPU exeception
- misaligned memory access
- virtual memory exceptions: page faults, TLB misse, protection violations
- traps: system calls, e.g., jumps into kernel

Exception classes:
- synchronous vs asynchronous, asynch caused by devices external to the CPU and memory and can be handled easily
- user requested vs coerced, user requested are predictable: treated as exceptions because they use the same mechanism that are used
to save and restore the state;
handled after the instruction has completed.
Coerced are caused by some HW event not under control of the program.
- user maskable vs user nonmaskable
- within vs between instructions
- resume vs terminate

For the exam study difference between pairs.

Invoking the interrupt handler:
- terminate exec of instr till PC-4
- store the PC in EPC register, exception program counter
- disable other interrupts and transfers control to a designated interrupt running in the kernel mode

In the cpu we have two modes: user-mode and kernel-mode (unstoppable).

precise interrupts
- easy handled

Exception handling in the 5-stage pipeline
