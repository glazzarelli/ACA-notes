%! Author = lazza
%! Date = 07/05/2022

\section{Explicit register renaming}\label{sec:register-renaming}
Tomasulo provides \textit{implicit} register renaming by the means of reservation stations or ROB\@.
Now we introduce \textit{explicit} register renaming: use a physical register file that is larger than the number of
registers specified by the ISA\@.

Idea: allocate a new physical destination register for every instruction that writes:
\begin{itemize}[noitemsep]
    \item removes WAW, WAR hazards
    \item allows out-of-order completion (like tomasulo)
    \item similar to SSA, Static Single Assignment transformation done the compiler
\end{itemize}

\paragraph{Mechanism} Keep a translation table:
\begin{itemize}[noitemsep]
    \item map ISA \textit{logical} register to physical register
    \item when a register is written, replace the map entry with new register from freelist
    \item physical register becomes free when not used by any active instructions
\end{itemize}

\paragraph{Unified physical register file}
\begin{itemize}
    \item rename all architectural (logical) registers into a single physical register file during decode, no register
    values read
    \item FUs read and write from single unified register file holding committed and temporary registers in execution
    \item commit only updates mapping of architectural register to physical register, no data movement
\end{itemize}

\paragraph{HW register renaming}
\begin{description}
    \item[Renaming map] simple data structure that supplies the physical register number of the register that
    currently corresponds to the requested architectural register
    \item[Instruction commit] update permanently the renaming table to indicate that the physical register holding
    the destination values corresponds to the actual architectural register
    \item[ROB] Use reorder buffer to enforce in-order commit
\end{description}

\subsection{Explicit renaming - Scoreboard}\label{subsec:explicit-renaming---scoreboard}
\subsubsection{Stages}
\begin{description}
    \item[Issue] decode instructions and check for structural hazards \textcolor{red}{and allocate new physical
    register for result}:
    \begin{itemize}[noitemsep]
        \item[-] instructions issued in program order (for hazard checking)
        \item[-] \textcolor{red}{don't issue if there are no free physical registers}
        \item[-] don't issue if there's a structural hazard
    \end{itemize}
    \item[Read operands] wait until there are no more hazards, then read operands.
    All real dependencies solved in this stage, since we wait for instructions to write data back.
    \item[Execution] operate on operands.
    The functional unit begins execution upon receiving operands.
    When the result is ready, it notifies the scoreboard.
    \item[Write result] finish execution
\end{description}
\textbf{Note:} \textcolor{red}{no check for WAR and WAW hazards.}

\subsection{Register renaming vs ROB}\label{subsec:register-renaming-vs-rob}
\begin{itemize}[noitemsep]
    \item[-] instruction commit simpler than with ROB
    \item[-] deallocating register more complex
    \item[-] dynamic mapping or architectural to physical registers complicates design and debugging.
\end{itemize}


